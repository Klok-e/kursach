\renewcommand\bibname{ПЕРЕЛІК ДЖЕРЕЛ ПОСИЛАННЯ}
\begin{thebibliography}{9}
    \bibitem{latex:friends}
    What are TeX and its friends? [Електронний ресурс] – Режим доступу до ресурсу: https://www.ctan.org/tex.

    \bibitem{latex:oss-devs-latex}
    Gaudeul A. Do Open Source Developers Respond to Competition? The LATEX Case Study / Alex Gaudeul. // Review of Network Economics. – 2007.

    \bibitem{webhooks:define}
    Webhooks [Електронний ресурс]. – 2019. – Режим доступу до ресурсу: https://developer.atlassian.com/server/jira/platform/webhooks/.

    \bibitem{webhooks:good}
    Lindsay J. Web hooks to revolutionize the web [Електронний ресурс] / Jeff Lindsay. – 2007. – Режим доступу до ресурсу: https://web.archive.org/web/20180630220036/http://progrium.com/blog/2007/05/03/web-hooks-to-revolutionize-the-web/.
    
    \bibitem{transformers:repo}
    State-of-the-art Natural Language Processing for Jax, PyTorch and TensorFlow [Електронний ресурс] – Режим доступу до ресурсу: https://github.com/huggingface/transformers.

    \bibitem{flashtext:arxiv}
    Replace or Retrieve Keywords In Documents at Scale [Електронний ресурс]. – 2017. – Режим доступу до ресурсу: https://arxiv.org/abs/1711.00046.

    \bibitem{flashtext:repo}
    FlashText module [Електронний ресурс] – Режим доступу до ресурсу: https://github.com/vi3k6i5/flashtext.

    \bibitem{ahocorasik:wiki}
    Aho–Corasick algorithm [Електронний ресурс] – Режим доступу до ресурсу: https://en.wikipedia.org/wiki/Aho%E2%80%93Corasick_algorithm.

    \bibitem{attention}
    Weng L. Attention? Attention! [Електронний ресурс] / Lilian Weng // lilianweng.github.io/lil-log. – 2018. – Режим доступу до ресурсу: http://lilianweng.github.io/lil-log/2018/06/24/attention-attention.html.

    \bibitem{cortex-stuff}
    Hubel D. Receptive fields, binocular interaction and functional architecture in the cat’s visual cortex / D. Hubel, T. Wiesel. // The Journal of physiology. – 1962. – С. 106–154.

    \bibitem{cortex-equations}
    Hodgkin A. quantitative description of membrane current and its application to conduction and excitation in nerve / A. Hodgkin, A. Huxley. // The Journal of physiology. – 1952. – С. 500–544.

    \bibitem{rozenblatt}
    Rosenblatt F. The perceptron, a perceiving and recognizing automaton Project Para / Frank Rosenblatt. // Cornell Aeronautical Laboratory. – 1957.
\end{thebibliography}